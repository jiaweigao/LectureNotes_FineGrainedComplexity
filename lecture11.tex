\documentclass[letterpaper, 12pt]{article}
 
\usepackage{jiawei}
\usepackage{complexity}
\usepackage{tikz}
\usepackage[charter]{mathdesign} 

\headfoot{FGC Notes}{Lecture 11}
 
\newcommand{\Perm}{\mathbf{Perm}}
\newcommand{\CAPP}{\mathbf{CAPP}}
\newcommand{\PIT}{\mathbf{PIT}}
\newcommand{\Time}[1]{\mathsf{TIME}[#1]}
\newcommand{\NTime}[1]{\mathsf{NTIME}[#1]}
\newcommand{\Size}[1]{\mathsf{Size}(#1)}
\newcommand{\Prob}[2]{\mathrm{Prob}_{#1}[#2]}
 
\newcommand{\subexp}{2^{n^{o(1)}}}
\newcommand{\csat}{\textrm{Circuit-SAT}}
\newcommand{\ctaut}{\textrm{Circuit-TAUT}}
\newcommand{\taut}{\textrm{-TAUT}}

 
\begin{document}
	\title{FGC Lecture 11}
	\date{}
	\maketitle
	
\begin{theorem}
	If $\csat \in \P$, then $\EXP$ has hard functions (Size$(f) \geq 2^n /n$).
\end{theorem}

\begin{proof}
	If $\csat$(which is $\NP$-complete) is in $\P$, then $\PH$ collapses to $\P$. Therefore $\EH$ collapses to $\EXP$. Then $\EXP = \Sigma_3^\EXP$, which contains hard problems (by Kannan's Theorem).
\end{proof}

\begin{theorem}
	If $\csat \in \Time{\subexp}$, then $\NEXP \nsubseteq \Ppoly$.
\end{theorem}
\begin{proof}
	If $\EXP \nsubseteq \Ppoly$, then we've done.
	
	If $\EXP \subseteq \Ppoly$, by Meyer's Theorem, $\EXP = \Sigma_2^\P$.
	
	For $L \in \EXP$, $x \in L \Leftrightarrow \exists y_1 \forall y_2 R(x, y_1, y_2)$.
	
	By assumption, $\exists y_1 \forall y_2 R(x, y_1, y_2)$ is in time $\subexp$. So $L \in \NTime{\subexp}$. So $\Sigma_3^\P \subseteq \EXP \subseteq \NTime{\subexp}$.
	
	Scaling up, $\exists T(n) = n^{\omega(1)}$, s.t. $\Sigma_3^{T(n)} \subseteq \NEXP$.
	
	$\Sigma_3^{T(n)} \nsubseteq \Ppoly$, thus $\NEXP \nsubseteq \Ppoly$.
\end{proof}

\sep

\begin{theorem}[Williams]
	If $\csat \in \Time{|C|\cdot 2^n / n^{\omega(1)}}$, then $\NEXP \nsubseteq \Ppoly$.
\end{theorem}

We prove the nondeterministic version of this theorem: If $\ctaut \in \NTime{|C|\cdot 2^n / n^{\omega(1)}}$, then $\NEXP \nsubseteq \Ppoly$.

\begin{proof}
	Assume $\NEXP \subseteq \Ppoly$, and $\ctaut\in \NTime{|C|\cdot 2^n / n^{\omega(1)}}$.
	
	Let $L \in \NTime{2^n}$. We are going to save time in solving $L$, thus contradicting the nondeterministic time hierarchy theorem.
	
	$x \in L \Leftrightarrow \exists y, |y|=2^n, R(x,y)$. $R$ is computable in time $2^n$.
	
	Let $C_R$ be a locally computable circuit that computes $R$. By Fischer-Pippenger's Theorem, $|C_R| = O(2^n \cdot n)$. Then,
	
	$x \in L \Leftrightarrow \exists (y, g_1, \dots, g_{2^n \cdot n})$, each gate is correct from its inputs, and output is 1.
	
	This new relation has succinct witness if $x \in L$:
	
	$x \in L \Leftrightarrow \exists C'' \forall i, \left[ (op_i(C''(x, j(i)), C''(x,k(i))) = C''(x,i)) \wedge (C''(x, \mathrm{output}) = \mathrm{True})\right]$
	
	Circuit $C''$ is a succinct witness.
	
	Let $T_{C''}(i)$ be the problem to check $C''$ on $i$. Then
	
	$x \in L \Leftrightarrow \exists C'', T_{C''}(i)$ is a tautology.
	
	By assumption, ``$T_{C''}(i)$ is a tautology'' is in $\NTime{2^{n'}/ {n'}^{\omega(1)}}$, where $n' = n + \log n$, equals $\NTime{2^{n}/ {n}^{\omega(1)}}$.
	
	So $L \in \NTime{2^{n}/ {n}^{\omega(1)}}$. Thus $\NTime{2^n} \subseteq \NTime{2^{n}/ {n}^{\omega(1)}}$, contradicting Time Hierarchy Theorem.
\end{proof}

\sep

\begin{theorem}[Williams]
	$\forall$ depth $d$, $\exists \epsilon$, s.t. $\ACC_6$-SAT $\in \Time{2^{n - n^\epsilon}}$.
\end{theorem}

\begin{lemma}\label{lemma1}
	If $\mathscr{C}$ is a class of circuits so that $\mathscr{C}\taut \in \NTime{2^n /n^{\omega(1)}}$ and $\P \subseteq \mathscr{C}$, then $\ctaut \in \NTime{2^n/n^{\omega(1)}}$.
\end{lemma}

If we have a circuit class $\mathscr{C}$ so that $\P \subseteq \mathscr{C}$, then it's not so easy, so it's good.

If $\P \nsubseteq \mathscr{C}$, then the result for $\ctaut$ applies to $\mathscr{C}\taut$.

\begin{corollary}
	If $\mathscr{C}\taut \in \NTime{2^n/n^{\omega(1)}}$, then $\NEXP \nsubseteq \mathscr{C}$.
\end{corollary}

\begin{corollary}
	$\NEXP \nsubseteq \ACC_6$.
\end{corollary}

\begin{proof}[Proof of Lemma \ref{lemma1}]
	If $\P \subseteq \mathscr{C}$, then $\Ppoly \subseteq \mathscr{C}$.
	
	Let $D$ be an instance of $\ctaut$, and let $g_1, \dots , g_m$ be its gates. Each $g_i$ defines a subcircuit of $D$, i.e. for each $g_i$, $\exists C_i \in \mathscr{C}$ s.t. $\forall x, g_i(x) = C_i(x)$.
	
	Our algorithm first non-deterministically guesses each $C_i$ for $i = 1, \dots , m$. Then, it verifies
	\[\forall x, \left[C_i(x) = op_i(C_{j(i)}(x),C_{k(i)}(x) )\right]\]
	Because $C_i$, $C_{j(i)}$ and $C_{k(i)}$ are all in $\mathscr{C}$ ($\mathscr{C}$ is closed), the verification can be done by $\mathscr{C}\taut$.
	
	We do the verification once per gate $i$, thus it is polynomial in input size.
	
	Finally we verify $C_{\text{output}}(x)$ is tautology.
\end{proof}
\end{document}